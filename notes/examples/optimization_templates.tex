%	This is written by Zhiyang Ong to indicate guidelines that members of open source software and/or hardware projects shall follow.

%	The MIT License (MIT)

%	Copyright (c) <2014> <Zhiyang Ong>

%	Permission is hereby granted, free of charge, to any person obtaining a copy of this software and associated documentation files (the "Software"), to deal in the Software without restriction, including without limitation the rights to use, copy, modify, merge, publish, distribute, sublicense, and/or sell copies of the Software, and to permit persons to whom the Software is furnished to do so, subject to the following conditions:

%	The above copyright notice and this permission notice shall be included in all copies or substantial portions of the Software.

%	THE SOFTWARE IS PROVIDED "AS IS", WITHOUT WARRANTY OF ANY KIND, EXPRESS OR IMPLIED, INCLUDING BUT NOT LIMITED TO THE WARRANTIES OF MERCHANTABILITY, FITNESS FOR A PARTICULAR PURPOSE AND NONINFRINGEMENT. IN NO EVENT SHALL THE AUTHORS OR COPYRIGHT HOLDERS BE LIABLE FOR ANY CLAIM, DAMAGES OR OTHER LIABILITY, WHETHER IN AN ACTION OF CONTRACT, TORT OR OTHERWISE, ARISING FROM, OUT OF OR IN CONNECTION WITH THE SOFTWARE OR THE USE OR OTHER DEALINGS IN THE SOFTWARE.

%	Email address: echo "cukj -wb- 23wU4X5M589 TROJANS cqkH wiuz2y 0f Mw Stanford" | awk '{ sub("23wU4X5M589","F.d_c_b. ") sub("Stanford","d0mA1n"); print $5, $2, $8; for (i=1; i<=1; i++) print "6\b"; print $9, $7, $6 }' | sed y/kqcbuHwM62z/gnotrzadqmC/ | tr 'q' ' ' | tr -d [:cntrl:] | tr -d 'ir' | tr y "\n"

%%%%%%%%%%%%%%%%%%%%%%%%%%%%%%%%%%%%%%%%%%%%%%



%%%%%%%%%%%%%%%%%%%%%%%%%%%%%%%%%%%%%%%%%%%%%%
%	Preamble.
\documentclass[letter,12pt]{article}
%%%%%%%%%%%%%%%%%%%%%%%%%%%%%%%%%%%%%%%%%%%%%
%
%	Importing LaTeX source files, without quoting the ".tex" extension.
%
%%%%%%%%%%%%%%%%%%%%%%%%%%%%%%%%%%%%%%%%%%%%%

%%%%%%%%%%%%%%%%%%%%%%%%%%%%%%%%%%%%%%%%%%%%%
%	File containing the LaTeX preamble.
\input{./others/preamble_section}





% definition of new \LaTeX command for the citation: \cite{Cimatti08} and \cite{Barrett09}
% This allows mathematical/logic symbols to be typeset with the font ``Zapf Chancery'' in ``\LaTeX\ math mode''. To typeset symbols in such font, try: \mathpzc{ABCdef123}
\DeclareMathAlphabet{\mathpzc}{OT1}{pzc}{m}{it}

%%%%%%%%%%%%%%%%%%%%%%%%%%%%%%%%%%%%%%%%%%%%%
% Start of document
\begin{document}
\title{Trying Out Ways to Typeset Mathematical Formulation of Optimization Problems}
\date{\today}
\author{Zhiyang Ong \thanks{Email correspondence to: \href{mailto:ongz@acm.org}{\Email\ ongz@acm.org}}}
\maketitle


\begin{abstract} 
This is an attempt to try out ways to typeset mathematical formulation of optimization problems \cite{Cay2013}.
\end{abstract}


%%%%%%%%%%%%%%%%%%%%%%%%%%%%%%%%%%%%%%%%%
%	Create the table of contents
\tableofcontents
%	Start the numbering of chapters from 1, instead of 0.
%\setcounter{chapter}{1}
%	Increase the depth of each section in the Table of Contents to 4.
\setcounter{secnumdepth}{4}



%%%%%%%%%%%%%%%%%%%%%%%%%%%%%%%%%%%%%%%%%%%
\section*{Revision History}
\label{sec:RevisionHistory}
\addcontentsline{toc}{section}{Revision History}


Revision history: \vspace{-0.3cm}
\begin{enumerate} \itemsep -4pt
\item Version 1, December 1, 2020. Initial version of this document.
\end{enumerate}









%%%%%%%%%%%%%%%%%%%%%%%%%%%%%%%%%%%%%%%%%%%
\section{Attempting Solutions from \cite{Cay2013}}
\label{sec:AttemptingSolutionsFromCay2013}


The suggested implementations in \LaTeX\ are actually from \href{https://github.com/eda-ricercatore/SienaLaTeX/blob/master/notes/examples/from-other-peeps/optimization-templates/main.pdf}{John Hammersley}, \href{https://github.com/eda-ricercatore/SienaLaTeX/blob/master/notes/examples/from-other-peeps/optimization-templates/main.tex}{who provided the \LaTeX\ document}. I modified the author list of the hyperlinked \LaTeX\ document to reflect this. In addition, John Hammersley incorporated a suggestion from Vince Knight for typesetting mathematical optimization models in \LaTeX.






%%%%%%%%%%%%%%%%%%%%%%%%%%%%%%%%%%%%%%%%%%%
\subsection{Array-based Formulation from \cite{Cay2013}}
\label{ssec:ArrayBasedFormulationFromCay2013}

Using the array-based formulation from \cite{Cay2013}.

\begin{equation}
\begin{array}{rrclcl}
\displaystyle \min_{x} & \multicolumn{3}{l}{c^T x} \\
\textrm{s.t.} & A x & \leq & b \\
&\displaystyle \sum_{i=0}^{n} x_i & = & 1 \\
& x_j & \geq & 0 & & \forall j \in N \\
\end{array}
\end{equation}


Notes: \vspace{-0.3cm}
\begin{enumerate} \itemsep -4pt
\item ``As you see, first column in the array is used for `min' and `subject to'. Hence, all lines except first two start with `\&' symbol. Second, third and fourth columns are used to define constraints.''
\item ``Note that, using eqnarray instead of array may lead some spacing inconsistencies.''
\item ``Advantages: Clean and tidy output, works well even for long constraints / models''
\item ``Disadvantages: You need to add `$\backslash$displaystyle' every time you use a summation symbol''
\item ``(Tip: Adding $\backslash$everymath\{$\backslash$displaystyle\} before $\backslash$begin\{document\} is another option as noted in comments)''
\end{enumerate}


%%%%%%%%%%%%%%%%%%%%%%%%%%%%%%%%%%%%%%%%%%%
\subsection{Aligned Formulation from \cite{Cay2013}}
\label{ssec:AlignedFormulationFromCay2013}

Using the aligned formulation from \cite{Cay2013}.



\begin{equation}
\begin{aligned}
& \underset{x}{\text{min}}
& & c^T x \\
& \text{s.t.} & &  Ax \leq b_i \\
& & &  \sum_{i=1}^{n} x_i =1 \\
& & &  x_j, \; \forall j \in N. \\
\end{aligned}
\end{equation}



Notes: \vspace{-0.3cm}
\begin{enumerate} \itemsep -4pt
\item ``Advantages: No need for `$\backslash$displaystyle' ''
\item ``Disadvantages: Alignment is not flexible (everything is left-aligned), multi-column is not available (if you have a long objective function, there will be some problems)''
\end{enumerate}





%%%%%%%%%%%%%%%%%%%%%%%%%%%%%%%%%%%%%%%%%%%
\subsection{Matrix Formulation from \cite{Cay2013}}
\label{ssec:MatrixFormulationFromCay2013}

Using the matrix formulation from \cite{Cay2013}.



\begin{equation}
\begin{matrix}
\displaystyle \min_x & c^T x  \\
\textrm{s.t.} & A x & \leq & b  \\
& \displaystyle \sum_{i=1}^{n} x_i & = & 1  \\
& x_j & \geq & 0 & & \forall j \in N
\end{matrix}
\end{equation}


Notes: \vspace{-0.3cm}
\begin{enumerate} \itemsep -4pt
\item ``This has similar problems to Aligned method. If your constraints have similar size, then you may like the result. To me, it looks good for this example.''
\item ``Disadvantages: Again, you need to switch between style modes ($\backslash$displaystyle)''
\end{enumerate}





%%%%%%%%%%%%%%%%%%%%%%%%%%%%%%%%%%%%%%%%%%%
\subsection{Align Formulation from \cite{Cay2013}}
\label{ssec:AlignFormulationFromCay2013}

Using the align formulation from \cite{Cay2013}.


\begin{align*}
\min_x \quad c^T x \\
Ax &\leq b \\
\sum_{i=1}^n x_i &= 1\\
x_j &\geq 0 \quad \forall j \in N
\end{align*}




Notes: \vspace{-0.3cm}
\begin{enumerate} \itemsep -4pt
\item ``You can also get the same result if you use `aligned', but this one is a different approach. You can think align as an array with two columns, where first column is always right-aligned and second column is always left-aligned. I trimmed `s.t.' since it leads a dirty result.''
\item ``Advantages: Clean and easy-to-edit''
\item ``Disadvantages: Manual alignment of extra parts (for all, etc.), objective function may shift if you use long constraints''
\end{enumerate}


%%%%%%%%%%%%%%%%%%%%%%%%%%%%%%%%%%%%%%%%%%%
\subsection{Comparison of Methods from \cite{Cay2013}}
\label{ssec:ComparisonOfMethodsFromCay2013}



``There's no method that is simply the best, it's all about applying the one which works for your needs. If you already published some papers, you'll already have a method in your mind to do it. I tried some of them for my research notes, and I prefer the first method. Flexibility of the first method is the key for my models. But, in the end, all methods has some features that serve you better than others.'' \\


``Jesus Lago Garcia commented that they have a \LaTeX\ package for defining optimization problems named {\tt optidef}. Thanks for the package and letting us know Jesus.'' See \url{https://www.ctan.org/pkg/optidef} about {\tt optidef}.








%%%%%%%%%%%%%%%%%%%%%%%%%%%%%%%%%%%%%%%%%%%%%
%%%%%%%%%%%%%%%%%%%%%%%%%%%%%%%%%%%%%%%%%%%%%
%
%	End of document
%
%	Inserting references
%
%%%%%%%%%%%%%%%%%%%%%%%%%%%%%%%%%%%%%%%%%%%%%
%%%%%%%%%%%%%%%%%%%%%%%%%%%%%%%%%%%%%%%%%%%%%
%	Beginning of BACK MATTER: bibliography, indexes and colophon
%\backmatter

%%%%%%%%%%%%%%%%%%%%%%%%%%%%%%%%%%%%%%%%%%%%%
{\linespread{1}
\bibliographystyle{plain}
%\bibliography{/Users/zhiyang/Documents/ricerca/antipastobibtex/references}
%\bibliography{/Users/zhiyang/Documents/ricerca/lassi-bibtex/references}
\bibliography{/Users/zhiyang/Documents/ricerca/saag-bibtex/references}
}
%\bibliography{/data/research/antipastobibtex/references}
%%%%%%%%%%%%%%%%%%%%%%%%%%%%%%%%%%%%%%%%%%%%%
\end{document}