%	This is written by Zhiyang Ong as a template for writing reports.

%	The MIT License (MIT)

%	Copyright (c) <2014> <Zhiyang Ong>

%	Permission is hereby granted, free of charge, to any person obtaining a copy of this software and associated documentation files (the "Software"), to deal in the Software without restriction, including without limitation the rights to use, copy, modify, merge, publish, distribute, sublicense, and/or sell copies of the Software, and to permit persons to whom the Software is furnished to do so, subject to the following conditions:

%	The above copyright notice and this permission notice shall be included in all copies or substantial portions of the Software.

%	THE SOFTWARE IS PROVIDED "AS IS", WITHOUT WARRANTY OF ANY KIND, EXPRESS OR IMPLIED, INCLUDING BUT NOT LIMITED TO THE WARRANTIES OF MERCHANTABILITY, FITNESS FOR A PARTICULAR PURPOSE AND NONINFRINGEMENT. IN NO EVENT SHALL THE AUTHORS OR COPYRIGHT HOLDERS BE LIABLE FOR ANY CLAIM, DAMAGES OR OTHER LIABILITY, WHETHER IN AN ACTION OF CONTRACT, TORT OR OTHERWISE, ARISING FROM, OUT OF OR IN CONNECTION WITH THE SOFTWARE OR THE USE OR OTHER DEALINGS IN THE SOFTWARE.

%	Email address: echo "cukj -wb- 23wU4X5M589 TROJANS cqkH wiuz2y 0f Mw Stanford" | awk '{ sub("23wU4X5M589","F.d_c_b. ") sub("Stanford","d0mA1n"); print $5, $2, $8; print " "; for (i=1; i<=1; i++) print "6\b"; print $9, $7, $6 }' | sed y/kqcbuHwM62z/gnotrzadqmC/ | tr 'q' ' ' | tr -d "\n" | tr -d 'ir' | tr y "\n"

%%%%%%%%%%%%%%%%%%%%%%%%%%%%%%%%%%%%%%%%%%%%%%









%%%%%%%%%%%%%%%%%%%%%%%%%%%%%%%%%%%%%%%%%%%%%%
%	Preamble.
\documentclass[letterpaper,12pt]{report}
%%%%%%%%%%%%%%%%%%%%%%%%%%%%%%%%%%%%%%%%%%%%%
%
%	Importing LaTeX source files, without quoting the ".tex" extension.
%
%%%%%%%%%%%%%%%%%%%%%%%%%%%%%%%%%%%%%%%%%%%%%

% Commands for using the package for hyperlinks. Includes the package ``url''.
\usepackage[pdftex,
	pdftitle={Graphics and Color with LaTeX},
	pdfauthor={Patrick W Daly},
	pdfsubject={Importing images and use of color in LaTeX},
	pdfkeywords={LaTeX, graphics, color},
	pdfpagemode=UseOutlines,bookmarks, bookmarksopen,
	pdfstartview=FitH, colorlinks, linkcolor=blue, citecolor=blue, urlcolor=red,
]{hyperref}
\hypersetup{colorlinks, linkcolor=blue}


% Concatenate adjacent references together when typeset.
% That is, cite{ref1,ref2,ref3,ref4} can appear as [12-15], instead of [12] [13] [14] [15]
\usepackage{others/packages/cite}

% The marvosym package (Table 210, page 69) is for Communication Symbols, such as \Email, \fax, \FAX (Preferred), \Letter, \Mobilefone, and \Telefon; it also has the symbol for the Cross to represent Christianity, \Cross (Table 263, pp.80); it also has symbols for checked boxes, \Checkedbox, crossed boxes (boxes marked with a cross), \Crossedbox, bicycles, \Bicycle, clocks, \Clocklogo, the industry, \Industry, taking notes manually with pen/pencil and paper, \Writinghand, coffee, \Coffeecup, providing information or important note, \Info (Table 249, pp.76)... In addition, it has the symbols for the Euro (EU currency), \EUR (OK), \EURdig (OK), \EURtm, \EURcr
\usepackage{marvosym}

% LaTeX support for Metafont and MetaPost logos.
\usepackage{mflogo}

% The textcomp package is for Diacritics -- I wanna use the \textacutedbl symbol to represent a double quote (Table 28, pp.17), instead of using the generic ``curly'' double quotes from \LaTeX; however, when this symbol is used, I must force a character space to exist after the symbol by using the backslash followed by a character space. This package also provides the symbol for Copyleft, \textcopyleft, which is not available in LaTeX by default, and provides better looking symbols for: copyright, registered, and trademark (Table 33, pp.18). Also, it provides symbols for: \textcelsius, \textmho, \textmu, \textohm (Table 201, pp.67). It also provides symbols for Genealogical Symbols (Table 253, pp78), such as \textborn, \textdivorced, \textmarried, \textdied, and \textleaf (symbol of a leaf)... Its symbol for the Euro, EU currency, is \texteuro
\usepackage{textcomp}

% The eurosym package has the symbols for the Euro (EU currency), \geneuro, \geneuronarrow, \geneurowide, \officialeuro (GOOD)
\usepackage{eurosym}

% For better typesetting of mathematical expressions, from the American Mathematical Society (AMS).
\usepackage{amsmath}
% For better typesetting of mathematical expressions, from the American Mathematical Society (AMS). This package includes mathematical symbols for the ``amsmath'' package.
\usepackage{amssymb}
% For better typesetting of mathematical proofs (for theorems and colloraries), from the American Mathematical Society (AMS).
\usepackage{amsthm}
%	Create definitions for new theorems, axioms, colloraries.
	\newtheorem{theorem}{Theorem}[chapter]
	\newtheorem{axiom}{Axiom}[chapter]
	\newtheorem{corollary}{Corollary}[chapter]
	\newtheorem{lemma}{Lemma}[chapter]
	\newtheorem{Rule}{Rule}[chapter]
	\newtheorem{law}{Law}[chapter]
	\newtheorem{principle}{Principle}[chapter]
% To change the style of newly defined theorems.
%		\usepackage{theorem}







%%%%%%%%%%%%%%%%%%%%%%%%%%%%%%%%%%%%%%%%%%%%%
%	File containing the LaTeX preamble.
%\input{./others/preamble_clean}
%	Enable the use of right-sided cases.
%\usepackage{others/packages/mathtools}
\usepackage{./others/packages/mathtools}
%\usepackage{extarrows}
%	-	-	-	-	-	-	-	-	-	-	-	-	-	-	-	-	-
%	Importing packages to typeset algorithms and pseudocode.
%\usepackage{clrscode3e}
%	Already imported in the preamble.
%\usepackage{clrscode3e}
%\usepackage{listings}
%	+ For the algorithmicx package, do not use: \usepackage{algorithmicx}
%		- Instead, import the package: algpseudocode.
%\usepackage{algpseudocode}
%\usepackage{./others/packages/algorithmicx}
\usepackage{./others/packages/algorithm}
\usepackage[compatible]{./others/packages/algpseudocode}

% To define multiple floats (figures and tables), with individual captions and labels, within one environment.
\usepackage{others/packages/subfig}

%	-	-	-	-	-	-	-	-	-	-	-	-	-	-	-	-	-
%	I have problems installing this LaTeX package, via the following commands.
%
%	latex algorithms.ins
%	latex algorithms.dtx
%
%	dyld: Library not loaded: /opt/local/lib/libssl.1.0.0.dylib
%	Referenced from: /opt/local/lib/libcurl.4.dylib
%	Reason: image not found
%\usepackage{algorithms}
%	-	-	-	-	-	-	-	-	-	-	-	-	-	-	-	-	-
% Alternative packages for typesetting algorithms.
%\usepackage{algorithm2e}




% The tipa package is for Phonetic Symbols.
%	I want to use the \textceltpal symbol to represent a single quote, instead of using the generic ``curly'' single quote from \LaTeX (Table 10, pp.10).
%	I want to use the \textcrd symbol to represent a ``d'' with a diacritic bar/stroke across it.
\usepackage{tipa}




%%%%%%%%%%%%%%%%%%%%%%%%%%%%%%%%%%%%%%%%%%%%%
%
%	Start of LaTeX document
%
%%%%%%%%%%%%%%%%%%%%%%%%%%%%%%%%%%%%%%%%%%%%%
\begin{document}

%%%%%%%%%%%%%%%%%%%%%%%%%%%%%%%%%%%%%%%%%
%	File containing a list of ``the 68 predefined internal colors of the {\tt dvips} PostScript driver'' \cite{Kopka04} 
%	This allows me to use any of these ``68 predefined internal colors''
\input{./others/list_of_colors.def}









%%%%%%%%%%%%%%%%%%%%%%%%%%%%%%%%%%%%%%%%%%%%%
%%%%%%%%%%%%%%%%%%%%%%%%%%%%%%%%%%%%%%%%%%%%%
%
%	Information for Top Matter / Title page
%
%%%%%%%%%%%%%%%%%%%%%%%%%%%%%%%%%%%%%%%%%%%%%
%%%%%%%%%%%%%%%%%%%%%%%%%%%%%%%%%%%%%%%%%%%%%
%	Beginning of FRONT MATTER: title page, table of contents and prefaces
%	Note that the \vspace{length} command does NOT work for the front matter
%\frontmatter
\title{\Huge \bf Title of the Report: Some Details about the Report}

%	Indicate the date of the report.
\date{\today}

\author{{\LARGE Name Surname}
\thanks{Email correspondence to: \href{mailto:email-id@domain.url}{\Email\ email-id@domain.url}}
\ \\
\vspace{-4.0in}
\ \\
\ \\
\ \\
{\bf \LARGE
	Name of Organization
	\vspace{0.1cm}} \\
\hline
\ \\
{\Large \sc Name of Group/Division} \\
\ \\
\ \\
\ \\
\ \\
\ \\
\vspace{2.0in}
\ \\
{\large \sc What is this report for?} \\
{\large It is for \dots} \\
{\large and BLAH \dots}
}

%	Create the title page.
\maketitle


%%%%%%%%%%%%%%%%%%%%%%%%%%%%%%%%%%%%%%%%%%%%%
%
%	Abstract

\begin{abstract} 
Insert abstract here. \\

More stuff to be included.
\end{abstract}

%%%%%%%%%%%%%%%%%%%%%%%%%%%%%%%%%%%%%%%%%%%%%
%%%%%%%%%%%%%%%%%%%%%%%%%%%%%%%%%%%%%%%%%%%%%

% Set the page numbering to lowercase Roman numerals
\pagenumbering{roman}
% Set the initial page number of the pages in the content section to be ``i''
\setcounter{page}{1}

%%%%%%%%%%%%%%%%%%%%%%%%%%%%%%%%%%%%%%%%%
%	Revision History
\input{./others/revision_history}





%%%%%%%%%%%%%%%%%%%%%%%%%%%%%%%%%%%%%%%%%
%	Create the table of contents
\tableofcontents
%	Start the numbering of chapters from 1, instead of 0.
%\setcounter{chapter}{1}
%	Increase the depth of each section in the Table of Contents to 4.
\setcounter{secnumdepth}{4}

%	List of Figures		=> Insert Here!!!
%	List of Tables			=> Insert Here!!!
%	List of To-Do Tasks	=> Insert Here!!!
%\listoftodos





%%%%%%%%%%%%%%%%%%%%%%%%%%%%%%%%%%%%%%%%%%%%%
%%%%%%%%%%%%%%%%%%%%%%%%%%%%%%%%%%%%%%%%%%%%%
\newpage
%\mainmatter

%	Body, or main section, of the document
%	Set the page numbering to normal (Arabic) numerals
\pagenumbering{arabic}
% Set the page numbers for the body of the document
\setcounter{page}{1}





%%%%%%%%%%%%%%%%%%%%%%%%%%%%%%%%%%%%%%%%%
%	Typesetting Text in LaTeX
%	This is written by Zhiyang Ong as a template for typesetting in LaTeX.

%	The MIT License (MIT)

%	Copyright (c) <2014> <Zhiyang Ong>

%	Permission is hereby granted, free of charge, to any person obtaining a copy of this software and associated documentation files (the "Software"), to deal in the Software without restriction, including without limitation the rights to use, copy, modify, merge, publish, distribute, sublicense, and/or sell copies of the Software, and to permit persons to whom the Software is furnished to do so, subject to the following conditions:

%	The above copyright notice and this permission notice shall be included in all copies or substantial portions of the Software.

%	THE SOFTWARE IS PROVIDED "AS IS", WITHOUT WARRANTY OF ANY KIND, EXPRESS OR IMPLIED, INCLUDING BUT NOT LIMITED TO THE WARRANTIES OF MERCHANTABILITY, FITNESS FOR A PARTICULAR PURPOSE AND NONINFRINGEMENT. IN NO EVENT SHALL THE AUTHORS OR COPYRIGHT HOLDERS BE LIABLE FOR ANY CLAIM, DAMAGES OR OTHER LIABILITY, WHETHER IN AN ACTION OF CONTRACT, TORT OR OTHERWISE, ARISING FROM, OUT OF OR IN CONNECTION WITH THE SOFTWARE OR THE USE OR OTHER DEALINGS IN THE SOFTWARE.

%	Email address: echo "cukj -wb- 23wU4X5M589 TROJANS cqkH wiuz2y 0f Mw Stanford" | awk '{ sub("23wU4X5M589","F.d_c_b. ") sub("Stanford","d0mA1n"); print $5, $2, $8; for (i=1; i<=1; i++) print "6\b"; print $9, $7, $6 }' | sed y/kqcbuHwM62z/gnotrzadqmC/ | tr 'q' ' ' | tr -d [:cntrl:] | tr -d 'ir' | tr y "\n"

%%%%%%%%%%%%%%%%%%%%%%%%%%%%%%%%%%%%%%%%%%%%%%



%%%%%%%%%%%%%%%%%%%%%%%%%%%%%%%%%%%%%%%%%%%
\chapter{Text}
\label{chp:Text}

There are a significant amount of references for helping people to learn \LaTeX \cite{Voss2011,vanDongen2012,Syropoulos2003,Raymond2004,Mittelbach2004,Lamport1994,Krishnan2003,Krantz2001,Kottwitz2011,Koranne2011,Kopka2004,Knuth1999,Hoenig1998,Higham1998,Haralambous2007,Griffiths1997,Gratzer2007,Goossens2007,Goossens1999,Goossens1997,Diller1999,Bindner2011,Berry2009,UITCambridge2011,Scharrer2011,Pakin2008,Cormen2010,Syropoulos2004,Hamalainen2006} and related information/technologies. \\


In this chapter, I will provide some templates for referencing, templates for {\sc Bib}\TeX\ entries, indicate some common \LaTeX\ symbols, usage of colors in \LaTeX, and miscellaneous details. \\


Random macros from my \LaTeX-specific IDE (or text editor): \vspace{-0.3cm}
\begin{enumerate} \itemsep -4pt
\item $\backslash\backslash\ \backslash$rule\{6in\}\{.1pt\}			%	\\ \rule{6in}{.1pt}
\item \href{mailto:emailid@domain.com}{emailid@domain.com}		%	\href{mailto:emailid@domain.com}{emailid@domain.com}
\item Begin-end constructs (i.e., $\backslash$begin and $\backslash$end) for: \vspace{-0.3cm}
	\begin{enumerate} \itemsep -2pt
	\item quotation
	\item quote
	\item verbatim
	\item verse
	\end{enumerate}
\item Types of headings: \vspace{-0.3cm}
	\begin{enumerate} \itemsep -2pt
	\item $\backslash$chapter\{\}
	\item $\backslash$paragraph\{\}
	\item $\backslash$subparagraph\{\}
	\item $\backslash$section\{\}
	\item $\backslash$subsection\{\}
	\item $\backslash$subsubsection\{\}
	\end{enumerate}
\item To add an entry into the ``Table of Contents'' without it being numbered, try the following: \vspace{-0.3cm}
	\begin{enumerate} \itemsep -2pt
	\item $\backslash$addcontentsline\{toc\}\{section\}\{BLAH\}
	\item $\backslash$section$^{\ast}$\{BLAH\}
	\end{enumerate}
\item Insert/import content from another file: $\backslash$input\{RELATIVE PATHNAME\}
\item Import \LaTeX\ packages: $\backslash$usepackage\{\}
\item $\backslash$footnote\{\}
\item $\backslash$marginpar\{\}
\item $\mathcal{C}$
\item $\mathit{C}$: Caligraphy style font.
\item \underline{This is good.}: Underline text.
\item \texttt{This is a statement.} TypeWriter.
\item \textsf{This is a statement.} Sans Serif font.
\item \textsl{This is a statement.} Slanted font.
\item \emph{This is a statement.}
\item Types of labels: \vspace{-0.3cm}
	\begin{enumerate} \itemsep -2pt
	\item ``chp:'' for chapter
	\item ``sec:'' for section
	\item ``ssec:'' for subsection
	\item ``sssec:'' for subsubsection
	\item ``fig:'' for figure
	\item ``tab:'' for table
	\item ``eqn:'' for equation
	\item ``lst:'' for code listing
	\item ``defn:'' for definition
	\item ``thrm:'' for theorem
	\item ``lem:'' for lemma
	\item ``crly:'' for corollary
	\item ``prop:'' for proposition
	\item ``prf:'' for proof
	\item ``eg:'' for example
	\item ``rem:'' for remark
	\end{enumerate}
\end{enumerate}


An enumeration of items: \vspace{-0.3cm}
\begin{enumerate} \itemsep -4pt
\item Quite sparse enumeration: \vspace{-0.3cm}
	\begin{enumerate} \itemsep -2pt
	\item Sparse enumeration: \vspace{-0.2cm}
		\begin{enumerate} \itemsep -2pt
		\item Very sparse enumeration: \vspace{-0.1cm}
			\begin{enumerate} \itemsep -1pt
			\item Very, very sparse list: \vspace{-0.1cm}
				\begin{itemize} \itemsep -1pt
				\item Blah
				\end{itemize}
			\end{enumerate}
		\end{enumerate}
	\end{enumerate}
\item 
\item 
\item Inserting a horizontal line beneath this item in the list.
\\ \rule{6in}{.1pt}
\item 
\item 
\end{enumerate}



To change the style for an enumerated list, try: {\tt $\backslash$begin\{enumerate\}[new\_style]} \cite{Kopka2004}.


%\renewcommand{\labelenumii}{\Alph{enumi}.\arabic{enumii}}

For example, to use Roman numerals, period separated by Arabic numerals, enclosed in round brackets, instead of the standard numbering, try \cite{Klement2016}: \vspace{-0.3cm}
%\begin{enumerate}[\renewcommand{\labelenumii}{\Alph{enumi}.\arabic{enumii}}] \itemsep -4pt
\begin{enumerate}[label=(\Alph*.\arabic*)] \itemsep -4pt
\item My item 1.
\item My item 2.
\item My item 3.: \vspace{-0.3cm}
	\begin{enumerate} \itemsep -2pt
	\item This is the first case.
	\item This is the second case.
	\end{enumerate}
\item My item 4.
\item My item 5.
\item My item 6.
\item My item 7.
\end{enumerate}



List of items: \vspace{-0.3cm}
\begin{itemize} \itemsep -4pt
\item Blah
\end{itemize}

Description of items: \vspace{-0.3cm}
\begin{description} \itemsep -4pt
\item[Key] Sparse description: \vspace{-0.3cm}
	\begin{description} \itemsep -2pt
	\item[key] Another entry
	\end{description}
\end{description}



Commonly forgotten \LaTeX\ typesetting information: \vspace{-0.3cm}
\begin{enumerate} \itemsep -4pt
\item Turkish {\tt i}: dis{\i}nformation. Second {\tt i} is a dotless {\tt i}.
\item Accents, diacritics, or diacritical marks/points/signs: \vspace{-0.3cm}
	\begin{enumerate} \itemsep -2pt
	\item Accents, diacritics, or diacritical marks/points/signs cannot be added above and below a given letter.
	\item \textcrd
%	Requires \usepackage{combelow}.
	\item Bucure$\cb{s}$ti-Ilfov
%	Requires \usepackage{wsuipa}.
%Bucure$\diaunder[,|s]$ti-Ilfov
%	Need to install wsuipa package.
	\end{enumerate}
\item The @ symbol (at sign, at symbol, commercial at, or address sign) can be used without the mathematical mode/environment \cite{Kopka2004}.
\item special characters: \vspace{-0.3cm}
	\begin{enumerate} \itemsep -2pt
	\item underscores: \vspace{-0.2cm}
		\begin{enumerate} \itemsep -2pt
		\item cheat\_sheets
		\end{enumerate}
	\item To indicate ``-\--'', try: ``-\--'' (-$\backslash$-\--). This would avoid turning the ``-\--'' into ``--'' \cite[\S2.5.3, pp. 26--27]{Kopka2004}.
	\end{enumerate}
\item brackets: \vspace{-0.3cm}
	\begin{enumerate} \itemsep -2pt
	\item Use $[$duplicate$]$ or [duplicate], rather than $\backslash$[duplicate$\backslash$]. %\[duplicate\].
	\item Use (duplicate) like normal.
	\end{enumerate}
\end{enumerate}






If text is required to be in the uppercase or capitals, it can still be written as normal, but use the \LaTeX\ command {\tt $\backslash$uppercase} to turn the text within the braces or curly brackets into uppercase. An example is provided as follows: ``\uppercase{This is an Example of Text Turned into UpperCase}'' \cite[\S8.2.4, pp. 239]{Syropoulos2003}. Another method is to use the \LaTeX\ command {\tt $\backslash$MakeUppercase} \cite[\S Appendix G.1, pp. 512]{Kopka2004}, and an example is: ``\MakeUppercase{Another Example of Text Turned into UpperCase}'' \cite[\S6.8, pp. 47; \S23.2, pp. 212--213]{Greenwade2022} \cite[\S2.2.2, pp. 31; \S3.1.5, pp. 85--87; \S3.1.7, pp. 91; \S4.4.2, pp. 229; \S9.4.1, pp. 571]{Mittelbach2004} \cite[\S3.5, pp. 60]{Syropoulos2003} \cite[\S5, Changing Letter Case]{Ying20XY}. \\

Their dual \LaTeX\ commands are: {\tt $\backslash$lowercase} and {\tt $\backslash$MakeLowercase} \cite[\S Appendix G.1, pp. 512]{Kopka2004}. Examples for these are: ``\lowercase{This is an Example of Text Turned into LowerCase}'' \cite[\S8.2.4, pp. 239]{Syropoulos2003} and ``\MakeLowercase{Another Example of Text Turned into LowerCase}'' \cite[\S23.2, pp. 212--213]{Greenwade2022} \cite[\S2.2.6, pp. 37; \S3.1.5, pp. 85--87; \S7.3.1, pp. 341; \S9.4.1, pp. 571]{Mittelbach2004} \cite[\S3.5, pp. 60]{Syropoulos2003} \cite[\S5, Changing Letter Case]{Ying20XY}. \\




External links, especially for the World Wide Web (W.W.W.), can be added as follows: \vspace{-0.3cm}
\begin{enumerate} \itemsep -4pt
\item \faYoutube \href{https://www.youtube.com/watch?v=TpGMCQCWd4M}{System/Technology Co-Optimization}
\item \faLink\ \href{https://www.youtube.com/watch?v=TpGMCQCWd4M}{System/Technology Co-Optimization}
\item \faExternalLink*\ \href{https://www.youtube.com/watch?v=TpGMCQCWd4M}{System/Technology Co-Optimization}
\end{enumerate}







%%%%%%%%%%%%%%%%%%%%%%%%%%%%%%%%%%%%%%%%%
%	Referencing for LaTeX documents via BibTeX
\input{./latex-text/citation}

%%%%%%%%%%%%%%%%%%%%%%%%%%%%%%%%%%%%%%%%%
%	Common LaTeX symbols
%	This is written by Zhiyang Ong as a template for writing LaTeX symbols.

%	The MIT License (MIT)

%	Copyright (c) <2014> <Zhiyang Ong>

%	Permission is hereby granted, free of charge, to any person obtaining a copy of this software and associated documentation files (the "Software"), to deal in the Software without restriction, including without limitation the rights to use, copy, modify, merge, publish, distribute, sublicense, and/or sell copies of the Software, and to permit persons to whom the Software is furnished to do so, subject to the following conditions:

%	The above copyright notice and this permission notice shall be included in all copies or substantial portions of the Software.

%	THE SOFTWARE IS PROVIDED "AS IS", WITHOUT WARRANTY OF ANY KIND, EXPRESS OR IMPLIED, INCLUDING BUT NOT LIMITED TO THE WARRANTIES OF MERCHANTABILITY, FITNESS FOR A PARTICULAR PURPOSE AND NONINFRINGEMENT. IN NO EVENT SHALL THE AUTHORS OR COPYRIGHT HOLDERS BE LIABLE FOR ANY CLAIM, DAMAGES OR OTHER LIABILITY, WHETHER IN AN ACTION OF CONTRACT, TORT OR OTHERWISE, ARISING FROM, OUT OF OR IN CONNECTION WITH THE SOFTWARE OR THE USE OR OTHER DEALINGS IN THE SOFTWARE.

%	Email address: echo "cukj -wb- 23wU4X5M589 TROJANS cqkH wiuz2y 0f Mw Stanford" | awk '{ sub("23wU4X5M589","F.d_c_b. ") sub("Stanford","d0mA1n"); print $5, $2, $8; for (i=1; i<=1; i++) print "6\b"; print $9, $7, $6 }' | sed y/kqcbuHwM62z/gnotrzadqmC/ | tr 'q' ' ' | tr -d [:cntrl:] | tr -d 'ir' | tr y "\n"

%%%%%%%%%%%%%%%%%%%%%%%%%%%%%%%%%%%%%%%%%%%%%%



%%%%%%%%%%%%%%%%%%%%%%%%%%%%%%%%%%%%%%%%%%%
\section{Writing \LaTeX\ Symbols}
\label{sec:WritingLaTeXSymbols}


Symbols used to represent \LaTeX\ and related computer languages/technologies and concepts are: \vspace{-0.3cm}
\begin{enumerate} \itemsep -4pt
\item \LaTeX
\item \LaTeXe
\item {\sc Bib}\TeX\ (or B{\scriptsize IB}\TeX)
\item \AmS-\LaTeX
\item \MP
\item \MF
\item \texttrademark
\item \textregistered
\item To use the registered symbol as a superscript, avoid doing this in the math mode or in mathematical environments, since this will cause the registered symbol not to typeset properly. The following sequence of {\tt $\backslash$textsuperscript}{\tt $\backslash$textregistered} \LaTeX\ commands should be used instead, such as: {\it quectoSAT}\textsuperscript\textregistered\ solver. \vspace{-0.3cm}
	\begin{enumerate} \itemsep -2pt
	\item Use a backslash after it, just like the following symbols that can cause naturally occuring character space to disappear: \vspace{-0.2cm}
		\begin{enumerate} \itemsep -2pt
		\item \textsuperscript\textregistered\ needs space\dots Compared with \textsuperscript\textregistered\ needs space. \vspace{-0.1cm}
			\begin{enumerate} \itemsep -1pt
			\item nanoPlace II\textsuperscript\textregistered is far superior compared with picoPlace VI\textsuperscript\textregistered for detailed placement (without spacing).
			\item nanoPlace II\textsuperscript\textregistered\ is far superior compared with picoPlace VI\textsuperscript\textregistered\ for detailed placement (with spacing).
			\item When in doubt if a space is need, since this example does not indicate a need, use a character space anyway. It does not change the spacing by much.
			\end{enumerate}
		\item \LaTeX\ needs space\dots Compared with \LaTeX needs space.
		\item \LaTeXe\ needs space\dots Compared with \LaTeXe needs space.
		\item {\sc Bib}\TeX\ needs space\dots Compared with {\sc Bib}\TeX needs space.
		\item B{\scriptsize IB}\TeX\ needs space\dots Compared with B{\scriptsize IB}\TeX needs space.
		\item \MF\ needs space\dots Compared with \MF needs space.
		\item \MP\ needs space\dots Compared with \MP needs space.
		\item \texttrademark\ needs space\dots Compared with \texttrademark needs space.
		\item \textregistered\ needs space\dots Compared with \textregistered needs space.
		\item \copyright\ needs space\dots Compared with \copyright needs space.
		\item \textcopyleft\ needs space\dots Compared with \textcopyleft needs space.
		\item \AmS-\LaTeX\ needs space\dots Compared with \AmS-\LaTeX needs space.
		\item \officialeuro\ needs space\dots Compared with \officialeuro needs space.
		\end{enumerate}
	\end{enumerate}
\item \copyright
\item \textcopyleft
\end{enumerate}





Other symbols of interests: \vspace{-0.3cm}
\begin{enumerate} \itemsep -4pt
\item \officialeuro
\item ``$\backslash >$'': 
\item 
\end{enumerate}

















%%%%%%%%%%%%%%%%%%%%%%%%%%%%%%%%%%%%%%%%%
%	Coloring in LaTeX documents.
\input{./latex-text/colors}










%%%%%%%%%%%%%%%%%%%%%%%%%%%%%%%%%%%%%%%%%
%	Typesetting Mathematics
%	This is written by Zhiyang Ong as a template for typesetting mathematics in LaTeX.

%	The MIT License (MIT)

%	Copyright (c) <2014> <Zhiyang Ong>

%	Permission is hereby granted, free of charge, to any person obtaining a copy of this software and associated documentation files (the "Software"), to deal in the Software without restriction, including without limitation the rights to use, copy, modify, merge, publish, distribute, sublicense, and/or sell copies of the Software, and to permit persons to whom the Software is furnished to do so, subject to the following conditions:

%	The above copyright notice and this permission notice shall be included in all copies or substantial portions of the Software.

%	THE SOFTWARE IS PROVIDED "AS IS", WITHOUT WARRANTY OF ANY KIND, EXPRESS OR IMPLIED, INCLUDING BUT NOT LIMITED TO THE WARRANTIES OF MERCHANTABILITY, FITNESS FOR A PARTICULAR PURPOSE AND NONINFRINGEMENT. IN NO EVENT SHALL THE AUTHORS OR COPYRIGHT HOLDERS BE LIABLE FOR ANY CLAIM, DAMAGES OR OTHER LIABILITY, WHETHER IN AN ACTION OF CONTRACT, TORT OR OTHERWISE, ARISING FROM, OUT OF OR IN CONNECTION WITH THE SOFTWARE OR THE USE OR OTHER DEALINGS IN THE SOFTWARE.

%	Email address: echo "cukj -wb- 23wU4X5M589 TROJANS cqkH wiuz2y 0f Mw Stanford" | awk '{ sub("23wU4X5M589","F.d_c_b. ") sub("Stanford","d0mA1n"); print $5, $2, $8; print " "; for (i=1; i<=1; i++) print "6\b"; print $9, $7, $6 }' | sed y/kqcbuHwM62z/gnotrzadqmC/ | tr 'q' ' ' | tr -d "\n" | tr -d 'ir' | tr y "\n"

%%%%%%%%%%%%%%%%%%%%%%%%%%%%%%%%%%%%%%%%%%%%%%



%%%%%%%%%%%%%%%%%%%%%%%%%%%%%%%%%%%%%%%%%%%
\chapter{Mathematics}
\label{chp:Mathematics}


%%%%%%%%%%%%%%%%%%%%%%%%%%%%%%%%%%%%%%%%%%%
%\section{Mathematics}
%\label{chp:Mathematics}


Math symbols that I use frequently: \vspace{-0.3cm}
\begin{enumerate} \itemsep -4pt
\item $\mathbb{N}$
\item $\displaystyle\sum^{i = 1}_{n}$
\item $f(x) = \displaystyle\lim_{n \rightarrow \infty} \frac{f(x)}{g(x)}$
\item $\varnothing$
\item $q$
\item $\varepsilon\kappa\alpha\beta\eta$
%		The following do not work.
%			$\varepsilon \kappa {\'{\alpha}} \beta \eta$
%			$\varepsilon\kappa{\'{\alpha}}\beta\eta$
%			\varepsilon\kappa\alpha\beta\eta
%			$\varepsilon\kappa${\|{$\alpha$}}$\beta\eta$
\item {\'{a}}$\acute{\alpha}$
%		The following do not work.
%			{\'{a}}{\'{$\alpha$}}
%			{\'{a}}${\'{\alpha}}$
%			{\'{a}}$\'\alpha$
\end{enumerate}

A $3 \times 3$ matrix:
$\left(
\begin{array}{ccc}
	11 & 12 & 13 \\
	21 & 22 & 23 \\
	31 & 32 & 33
\end{array}
\right)$
\ \\
\ \\

Here is an equation:
\begin{equation}
\label{eqn:myeqnexample}
\iint_{\Sigma} \nabla \times \mathbf{F} \cdot \mathrm{d}\mathbf{\Sigma} = \oint_{\partial\Sigma} \mathbf{F} \cdot \mathrm{d} \mathbf{r}.
\end{equation}
\ \\
\ \\

Here is an equation that is not numbered.
\begin{equation*}
\nabla \times \mathbf{E} = -\frac{\partial \mathbf{B}} {\partial t}
\end{equation*}



Here is the set of Maxwell's equations that is numbered.
\begin{gather}
	\nabla \cdot \mathbf{E} = \frac {\rho} {\varepsilon_0} \\
	\nabla \cdot \mathbf{B} = 0 \\
	\nabla \times \mathbf{E} = -\frac{\partial \mathbf{B}} {\partial t} \\
	\nabla \times \mathbf{B} = \mu_0\left(\mathbf{J} + \varepsilon_0 \frac{\partial \mathbf{E}} {\partial t} \right)
\end{gather}


\begin{gather*}
	{\rm minimize \displaystyle\sum^{c}_{i = 1} c_{i} \cdot x_{i}} \\	%	objective function defined mathematically	\\
	\underline{x} \in S \\
	{\rm subject\ to:} \\
	%	constraints	\\
	x_{1} + x_{4} = 0 \\
	x_{3} + 7 \cdot x_{4} + 2\cdot x_{9} = 0
\end{gather*}


\begin{equation}
\label{eqn:caseenv}
f(n) = 
	\begin{cases}
	case-1 &: \mathrm{n\ is\ odd} \\
	case-2 &: \mathrm{n\ is\ even} \\
	\end{cases}
\end{equation}

\begin{proof}
This is a proof for BLAH \dots
\end{proof}




\begin{theorem}{TITLE of theorem.}
My theorem is\dots
\end{theorem}



\begin{axiom}{TITLE of axiom.}
Blah\dots
\end{axiom}



Cases of putting a bracket/parenthesis on the right side of the equation.
\begin{gather*}
	\left.\begin{aligned}
	B'&=-\partial \times E,\\
	E'&=\partial \times B - 4\pi j,
	\end{aligned}
	\right\}
	\quad\text{Maxwell's equations}
\end{gather*}


Cases of putting a bracket/parenthesis on the right side of the equation.\\
$\begin{rcases*}
	E = m c^2 & foo \\
	\int x-3\, dx & barbaz
\end{rcases*} y=f(x)$
\ \\
\ \\

Labeling an arrow: $\xrightarrow{ewq}$. \\






Symbols for mathematical logic: \vspace{-0.3cm}
\begin{enumerate} \itemsep -4pt
\item $\models$ or $\vDash$, entails
\item $\vdash$, infers/proves/concludes: \vspace{-0.3cm}
	\begin{enumerate} \itemsep -2pt
	\item E.g., used in sequents (general kind of assertion), such as $A_{1}, \dots., A_{m} \vdash B_{1}, \dots, B_{n}$, where the conditional formulas $A_{i}$ are the antecedents and the asserted formulas $B_{j}$ are the succedents or consequents.
	\end{enumerate}
\item $\Rightarrow$, or $\Longrightarrow$, implies
\item $\land$ or $\bigwedge$, conjunction, AND: \vspace{-0.3cm}
	\begin{enumerate} \itemsep -2pt
	\item $A\land B$
	\end{enumerate}
\item $\lor$ or $\bigvee$, disjunction, OR: \vspace{-0.3cm}
	\begin{enumerate} \itemsep -2pt
	\item $A\lor B$ is true if ${\displaystyle A}$ is true, or if ${\displaystyle B}$ is true, or if both ${\displaystyle A}$ and ${\displaystyle B}$ are true.
	\end{enumerate}
\item $p \rightarrow q$: \vspace{-0.3cm}
	\begin{enumerate} \itemsep -2pt
	\item $p$ is the antecedent, and $q$ is the consequent: \vspace{-0.2cm}
		\begin{enumerate} \itemsep -2pt
		\item $p$ is also called the protasis
		\end{enumerate}
	\item material implication, or simply implication, if $p$ \dots then $q$, IMPLY: \vspace{-0.2cm}
		\begin{enumerate} \itemsep -2pt
		\item conditional statement
		\item or $\neg p \lor q$: \vspace{-0.1cm}
			\begin{enumerate} \itemsep -1pt
			\item By commutativity, we have: $q \lor \neg p$.
			\item By double negation, we have: $\neg \neg q \lor \neg p$.
			\end{enumerate}
		\item or, $\neg q \rightarrow \neg p$
		\item contrapositive statement: \vspace{-0.1cm}
			\begin{enumerate} \itemsep -1pt
			\item $\neg q \longrightarrow \neg p$
			\item if not $q$ then not $p$
			\item reversal and negation of both statements
			\end{enumerate}
		\item inverse statement: \vspace{-0.1cm}
			\begin{enumerate} \itemsep -1pt
			\item $\neg p \longrightarrow \neg q$
			\item if not $p$ then not $q$
			\item negation of both statements
			\end{enumerate}
		\item negation statement: \vspace{-0.1cm}
			\begin{enumerate} \itemsep -1pt
			\item $\neg (p \longrightarrow q)$
			\item or, $p \land \neg q$
			\item includes contrapositive statement
			\item contradicts the implication
			\end{enumerate}
		\end{enumerate}
	\item logical connectives include: \vspace{-0.2cm}
		\begin{enumerate} \itemsep -2pt
		\item $\neg p$, negation, NOT, inverter
		\item $p \land q$
		\item $p \lor q$
		\item $p \rightarrow q$
		\item $p \leftrightarrow q$, $p \longleftrightarrow q$, biconditional, $p$ if and only if $q$, XNOR: \vspace{-0.1cm}
			\begin{enumerate} \itemsep -1pt
			\item $(p \rightarrow q) \land (q \rightarrow p)$ = $(p \land q) \lor (\neg p \land \neg q)$
			\item logic equality
			\item logical biconditional
			\item material biconditional
			\end{enumerate}
		\item $p \leftarrow q$, $p \longleftarrow q$, converse implication, \dots if: \vspace{-0.1cm}
			\begin{enumerate} \itemsep -1pt
			\item $p \leftarrow q$ = $q \rightarrow p$.
			\item if $q$ then $p$
			\item reversal of both statements
			\end{enumerate}
		\item $p \uparrow q$, alternative denial, not both, NAND: \vspace{-0.1cm}
			\begin{enumerate} \itemsep -1pt
			\item or $\neg (p \land q)$, or $\neg p \lor \neg q$
			\end{enumerate}
		\item $p \downarrow q$, joint denial, neither\dots nor, NOR: \vspace{-0.1cm}
			\begin{enumerate} \itemsep -1pt
			\item or $\neg (p \lor q) \land (\neg p \land \neg q)$
			\end{enumerate}
		\item $\rightarrow/$, material nonimplication, NIMPLY
		\end{enumerate}
	\end{enumerate}
\end{enumerate}

Note that by transposition, or valid rule of replacement, $(p \rightarrow q) \Longleftrightarrow (\neg q \rightarrow \neg p)$. Alternatively, $(p \rightarrow q) \vdash (\neg q \rightarrow \neg p)$ or $\frac{p \rightarrow q}{\therefore\ \neg q \rightarrow \neg p}$. \\


\cite[Tables 185--187, pp. 100, in \S3]{Pakin2021} provide mathematical symbols for vector calculus, asymptotic notation for computational complexity or circuit/network complexity, and types of numbers (e.g., natural, real, and complex numbers).








%%%%%%%%%%%%%%%%%%%%%%%%%%%%%%%%%%%%%%%%%
%	Typesetting for Data Visualization
\input{./data_visualization/tables}
%	This is written by Zhiyang Ong as a template for inserting figures in LaTeX.

%	The MIT License (MIT)

%	Copyright (c) <2014> <Zhiyang Ong>

%	Permission is hereby granted, free of charge, to any person obtaining a copy of this software and associated documentation files (the "Software"), to deal in the Software without restriction, including without limitation the rights to use, copy, modify, merge, publish, distribute, sublicense, and/or sell copies of the Software, and to permit persons to whom the Software is furnished to do so, subject to the following conditions:

%	The above copyright notice and this permission notice shall be included in all copies or substantial portions of the Software.

%	THE SOFTWARE IS PROVIDED "AS IS", WITHOUT WARRANTY OF ANY KIND, EXPRESS OR IMPLIED, INCLUDING BUT NOT LIMITED TO THE WARRANTIES OF MERCHANTABILITY, FITNESS FOR A PARTICULAR PURPOSE AND NONINFRINGEMENT. IN NO EVENT SHALL THE AUTHORS OR COPYRIGHT HOLDERS BE LIABLE FOR ANY CLAIM, DAMAGES OR OTHER LIABILITY, WHETHER IN AN ACTION OF CONTRACT, TORT OR OTHERWISE, ARISING FROM, OUT OF OR IN CONNECTION WITH THE SOFTWARE OR THE USE OR OTHER DEALINGS IN THE SOFTWARE.

%	Email address: echo "cukj -wb- 23wU4X5M589 TROJANS cqkH wiuz2y 0f Mw Stanford" | awk '{ sub("23wU4X5M589","F.d_c_b. ") sub("Stanford","d0mA1n"); print $5, $2, $8; for (i=1; i<=1; i++) print "6\b"; print $9, $7, $6 }' | sed y/kqcbuHwM62z/gnotrzadqmC/ | tr 'q' ' ' | tr -d [:cntrl:] | tr -d 'ir' | tr y "\n"

%%%%%%%%%%%%%%%%%%%%%%%%%%%%%%%%%%%%%%%%%%%%%%



%%%%%%%%%%%%%%%%%%%%%%%%%%%%%%%%%%%%%%%%%%%
\chapter{Figures}
\label{chp:Figures}

A template for inserting figures is shown in Figures \ref{fig:MyFigure1}, \ref{fig:MyFigure2}, \ref{fig:MyFigure3}, and \ref{fig:MyFigure4}. \\

I have used the {\tt $\backslash$clearpage} command to clear the remanding part of the first page for this section (\S\ref{chp:Figures}), and insert the remaining figures and text in subsequent pages. If the last three figures (Figures  \ref{fig:MyFigure3} and \ref{fig:MyFigure4}) are reordered to the following order, Figures \ref{fig:MyFigure4} and \ref{fig:MyFigure3}, the effects of the {\tt $\backslash$clearpage} command would be more evident.


\begin{figure}[h]
\centering 
\includegraphics[height=1.5in]{./pics/short}
\caption{Caption for my figure1}
\label{fig:MyFigure1}
\end{figure}

\begin{figure}[h]
\centering 
\includegraphics[height=1.5in]{./pics/small}
\caption{Caption for my figure2}
\label{fig:MyFigure2}
\end{figure}

%	Clear the remanding part of the page, and insert the remaining figures and text in subsequent pages.
\clearpage

\begin{figure}[h]
\centering 
\includegraphics[width=6in]{./pics/my_figure}
\caption{Caption for my figure3}
\label{fig:MyFigure3}
\end{figure}






\begin{figure}[h]
\centering 
\includegraphics[height=1.5in]{./pics/trim}
\caption{Caption for my figure4}
\label{fig:MyFigure4}
\end{figure}

















%%%%%%%%%%%%%%%%%%%%%%%%%%%%%%%%%%%%%%%%%
%	Typesetting Algorithms
\input{./algor/algorithms_algorithmicx}


%%%%%%%%%%%%%%%%%%%%%%%%%%%%%%%%%%%%%%%%%
%	Bibliography
%\input{./others/bibliography}




%%%%%%%%%%%%%%%%%%%%%%%%%%%%%%%%%%%%%%%%%%%%%
%%%%%%%%%%%%%%%%%%%%%%%%%%%%%%%%%%%%%%%%%%%%%
%
%	End of document
%
%	Inserting references
%
%%%%%%%%%%%%%%%%%%%%%%%%%%%%%%%%%%%%%%%%%%%%%
%%%%%%%%%%%%%%%%%%%%%%%%%%%%%%%%%%%%%%%%%%%%%
%	Beginning of BACK MATTER: bibliography, indexes and colophon
%\backmatter
\appendix

{\linespread{1}
\bibliographystyle{plain}
\bibliography{./references/references}
\addcontentsline{toc}{chapter}{Bibliography}
}
\end{document}