%	This is written by Zhiyang Ong as a template for typesetting in LaTeX.

%	The MIT License (MIT)

%	Copyright (c) <2014> <Zhiyang Ong>

%	Permission is hereby granted, free of charge, to any person obtaining a copy of this software and associated documentation files (the "Software"), to deal in the Software without restriction, including without limitation the rights to use, copy, modify, merge, publish, distribute, sublicense, and/or sell copies of the Software, and to permit persons to whom the Software is furnished to do so, subject to the following conditions:

%	The above copyright notice and this permission notice shall be included in all copies or substantial portions of the Software.

%	THE SOFTWARE IS PROVIDED "AS IS", WITHOUT WARRANTY OF ANY KIND, EXPRESS OR IMPLIED, INCLUDING BUT NOT LIMITED TO THE WARRANTIES OF MERCHANTABILITY, FITNESS FOR A PARTICULAR PURPOSE AND NONINFRINGEMENT. IN NO EVENT SHALL THE AUTHORS OR COPYRIGHT HOLDERS BE LIABLE FOR ANY CLAIM, DAMAGES OR OTHER LIABILITY, WHETHER IN AN ACTION OF CONTRACT, TORT OR OTHERWISE, ARISING FROM, OUT OF OR IN CONNECTION WITH THE SOFTWARE OR THE USE OR OTHER DEALINGS IN THE SOFTWARE.

%	Email address: echo "cukj -wb- 23wU4X5M589 TROJANS cqkH wiuz2y 0f Mw Stanford" | awk '{ sub("23wU4X5M589","F.d_c_b. ") sub("Stanford","d0mA1n"); print $5, $2, $8; for (i=1; i<=1; i++) print "6\b"; print $9, $7, $6 }' | sed y/kqcbuHwM62z/gnotrzadqmC/ | tr 'q' ' ' | tr -d [:cntrl:] | tr -d 'ir' | tr y "\n"

%%%%%%%%%%%%%%%%%%%%%%%%%%%%%%%%%%%%%%%%%%%%%%



%%%%%%%%%%%%%%%%%%%%%%%%%%%%%%%%%%%%%%%%%%%
\chapter{Text}
\label{chp:Text}

There are a significant amount of references for helping people to learn \LaTeX \cite{Voss2011,vanDongen2012,Syropoulos2003,Raymond2004,Mittelbach2004,Lamport1994,Krishnan2003,Krantz2001,Kottwitz2011,Koranne2011,Kopka2004,Knuth1999,Hoenig1998,Higham1998,Haralambous2007,Griffiths1997,Gratzer2007,Goossens2007,Goossens1999,Goossens1997,Diller1999,Bindner2011,Berry2009,UITCambridge2011,Scharrer2011,Pakin2008,Cormen2010,Syropoulos2004,Hamalainen2006} and related information/technologies. \\


In this chapter, I will provide some templates for referencing, templates for {\sc Bib}\TeX\ entries, indicate some common \LaTeX\ symbols, usage of colors in \LaTeX, and miscellaneous details. \\


Random macros from my \LaTeX-specific IDE (or text editor): \vspace{-0.3cm}
\begin{enumerate} \itemsep -4pt
\item $\backslash\backslash\ \backslash$rule\{6in\}\{.1pt\}			%	\\ \rule{6in}{.1pt}
\item \href{mailto:emailid@domain.com}{emailid@domain.com}		%	\href{mailto:emailid@domain.com}{emailid@domain.com}
\item Begin-end constructs (i.e., $\backslash$begin and $\backslash$end) for: \vspace{-0.3cm}
	\begin{enumerate} \itemsep -2pt
	\item quotation
	\item quote
	\item verbatim
	\item verse
	\end{enumerate}
\item Types of headings: \vspace{-0.3cm}
	\begin{enumerate} \itemsep -2pt
	\item $\backslash$chapter\{\}
	\item $\backslash$paragraph\{\}
	\item $\backslash$subparagraph\{\}
	\item $\backslash$section\{\}
	\item $\backslash$subsection\{\}
	\item $\backslash$subsubsection\{\}
	\end{enumerate}
\item To add an entry into the ``Table of Contents'' without it being numbered, try the following: \vspace{-0.3cm}
	\begin{enumerate} \itemsep -2pt
	\item $\backslash$addcontentsline\{toc\}\{section\}\{BLAH\}
	\item $\backslash$section$^{\ast}$\{BLAH\}
	\end{enumerate}
\item Insert/import content from another file: $\backslash$input\{RELATIVE PATHNAME\}
\item Import \LaTeX\ packages: $\backslash$usepackage\{\}
\item $\backslash$footnote\{\}
\item $\backslash$marginpar\{\}
\item $\mathcal{C}$
\item $\mathit{C}$: Caligraphy style font.
\item \underline{This is good.}: Underline text.
\item \texttt{This is a statement.} TypeWriter.
\item \textsf{This is a statement.} Sans Serif font.
\item \textsl{This is a statement.} Slanted font.
\item \emph{This is a statement.}
\item Types of labels: \vspace{-0.3cm}
	\begin{enumerate} \itemsep -2pt
	\item ``chp:'' for chapter
	\item ``sec:'' for section
	\item ``ssec:'' for subsection
	\item ``sssec:'' for subsubsection
	\item ``fig:'' for figure
	\item ``tab:'' for table
	\item ``eqn:'' for equation
	\item ``lst:'' for code listing
	\item ``defn:'' for definition
	\item ``thrm:'' for theorem
	\item ``lem:'' for lemma
	\item ``crly:'' for corollary
	\item ``prop:'' for proposition
	\item ``prf:'' for proof
	\item ``eg:'' for example
	\item ``rem:'' for remark
	\end{enumerate}
\end{enumerate}


An enumeration of items: \vspace{-0.3cm}
\begin{enumerate} \itemsep -4pt
\item Quite sparse enumeration: \vspace{-0.3cm}
	\begin{enumerate} \itemsep -2pt
	\item Sparse enumeration: \vspace{-0.2cm}
		\begin{enumerate} \itemsep -2pt
		\item Very sparse enumeration: \vspace{-0.1cm}
			\begin{enumerate} \itemsep -1pt
			\item Very, very sparse list: \vspace{-0.1cm}
				\begin{itemize} \itemsep -1pt
				\item Blah
				\end{itemize}
			\end{enumerate}
		\end{enumerate}
	\end{enumerate}
\item 
\item 
\item Inserting a horizontal line beneath this item in the list.
\\ \rule{6in}{.1pt}
\item 
\item 
\end{enumerate}



To change the style for an enumerated list, try: {\tt $\backslash$begin\{enumerate\}[new\_style]} \cite{Kopka2004}.

For example, to use Roman numerals instead of the standard numbering, try: \vspace{-0.3cm}
\begin{enumerate}[I-1] \itemsep -4pt
\item My item \#1
\item My item \#2
\item My item \#3
\item My item \#4
\item My item \#5
\item My item \#6
\item My item \#7
\end{enumerate}



List of items: \vspace{-0.3cm}
\begin{itemize} \itemsep -4pt
\item Blah
\end{itemize}

Description of items: \vspace{-0.3cm}
\begin{description} \itemsep -4pt
\item[Key] Sparse description: \vspace{-0.3cm}
	\begin{description} \itemsep -2pt
	\item[key] Another entry
	\end{description}
\end{description}



Commonly forgotten \LaTeX\ typesetting information: \vspace{-0.3cm}
\begin{enumerate} \itemsep -4pt
\item Turkish {\tt i}: dis{\i}nformation. Second {\tt i} is a dotless {\tt i}.
\item Accents, diacritics, or diacritical marks/points/signs: \vspace{-0.3cm}
	\begin{enumerate} \itemsep -2pt
	\item Accents, diacritics, or diacritical marks/points/signs cannot be added above and below a given letter.
	\item \textcrd
	\end{enumerate}
\item The @ symbol (at sign, at symbol, commercial at, or address sign) can be used without the mathematical mode/environment \cite{Kopka2004}.
\item special characters: \vspace{-0.3cm}
	\begin{enumerate} \itemsep -2pt
	\item underscores: \vspace{-0.2cm}
		\begin{enumerate} \itemsep -2pt
		\item cheat\_sheets
		\end{enumerate}
	\item To indicate ``-\--'', try: ``-\--'' (-$\backslash$-\--). This would avoid turning the ``-\--'' into ``--'' \cite[\S2.5.3, pp. 26--27]{Kopka2004}.
	\end{enumerate}
\item brackets: \vspace{-0.3cm}
	\begin{enumerate} \itemsep -2pt
	\item Use $[$duplicate$]$ or [duplicate], rather than $\backslash$[duplicate$\backslash$]. %\[duplicate\].
	\item Use (duplicate) like normal.
	\end{enumerate}
\item LaTeX-related symbols: \vspace{-0.3cm}
	\begin{enumerate} \itemsep -2pt
	\item \LaTeX.
	\item \LaTeXe.
	\item \AmS-\LaTeX.
	\item {\sc Bib}\TeX.
	\item \MF.
	\item \MP.
	\item \texttrademark.
	\item \textregistered.
	\item \copyright.
	\item \textcopyleft.
	\end{enumerate}
\end{enumerate}





%%%%%%%%%%%%%%%%%%%%%%%%%%%%%%%%%%%%%%%%%
%	Referencing for LaTeX documents via BibTeX
%	This is written by Zhiyang Ong as a template for writing text in LaTeX.

%	The MIT License (MIT)

%	Copyright (c) <2014> <Zhiyang Ong>

%	Permission is hereby granted, free of charge, to any person obtaining a copy of this software and associated documentation files (the "Software"), to deal in the Software without restriction, including without limitation the rights to use, copy, modify, merge, publish, distribute, sublicense, and/or sell copies of the Software, and to permit persons to whom the Software is furnished to do so, subject to the following conditions:

%	The above copyright notice and this permission notice shall be included in all copies or substantial portions of the Software.

%	THE SOFTWARE IS PROVIDED "AS IS", WITHOUT WARRANTY OF ANY KIND, EXPRESS OR IMPLIED, INCLUDING BUT NOT LIMITED TO THE WARRANTIES OF MERCHANTABILITY, FITNESS FOR A PARTICULAR PURPOSE AND NONINFRINGEMENT. IN NO EVENT SHALL THE AUTHORS OR COPYRIGHT HOLDERS BE LIABLE FOR ANY CLAIM, DAMAGES OR OTHER LIABILITY, WHETHER IN AN ACTION OF CONTRACT, TORT OR OTHERWISE, ARISING FROM, OUT OF OR IN CONNECTION WITH THE SOFTWARE OR THE USE OR OTHER DEALINGS IN THE SOFTWARE.

%	Email address: echo "cukj -wb- 23wU4X5M589 TROJANS cqkH wiuz2y 0f Mw Stanford" | awk '{ sub("23wU4X5M589","F.d_c_b. ") sub("Stanford","d0mA1n"); print $5, $2, $8; for (i=1; i<=1; i++) print "6\b"; print $9, $7, $6 }' | sed y/kqcbuHwM62z/gnotrzadqmC/ | tr 'q' ' ' | tr -d [:cntrl:] | tr -d 'ir' | tr y "\n"

%%%%%%%%%%%%%%%%%%%%%%%%%%%%%%%%%%%%%%%%%%%%%%



%%%%%%%%%%%%%%%%%%%%%%%%%%%%%%%%%%%%%%%%%%%
\section{Referencing Information}
\label{sec:RefInfo}

Here is how I can reference common resources: \vspace{-0.3cm}
\begin{enumerate} \itemsep -4pt
\item For online resources: \vspace{-0.3cm}
	\begin{enumerate} \itemsep -2pt
	\item Author, ``Title of web page,'' in {\it Title of Primary Web Site}, Name of Publisher/Organization/Individual, Address, Month Date, Year. Available online at: \url{http://www.webpage.url/}; last accessed on June 2, 2014.	% Available online at: \url{URL}; last accessed on June 2, 2014.
	\item Regarding entries for my {\sc Bib}\TeX\ database, insert the following to the ``howpublished'' field: Available online at: \url{http://www.webpage.url/}; June 11, 2012 was the last accessed date.	% Available online at: \url{URL}; June 11, 2012 was the last accessed date
	\end{enumerate}
\item DOI field in {\sc Bib}\TeX\ should be indicated as a URL: \url{http://dx.doi.org/DOI}.	% http://dx.doi.org/DOI
\item To enter a summary of a paper that I have written into a report, enter it as a section (or subsection or subsubsection) with the following ``fields'': \vspace{-0.3cm}
	\begin{enumerate} \itemsep -2pt
	\item In the title of the section, indicate the title of the paper and its abbreviation (i.e., its {\sc Bib}\TeX\ key).
	\item Terse summary: Summary of the paper in 2-3 lines.
	\item Not-so-concise summary and highlights. Summarize the publication in $\leq$ 2 pages. For publications that are not survey papers nor literature review, highlight the advantages and disadvantages of the described techniques/innovations. For survey papers nor literature review publications, summarize the primary publications that was mentioned in the survey/review.
	\item Other notes about the publication: Insert important figures and equations, among other details about the paper.
	\end{enumerate}
\item {\it BibDesk} only creates a folder for publications with non-empty author fields. Hence, when entering a {\sc Bib}\TeX\ into my {\sc Bib}\TeX\ database, enter the names of the editors into the {\tt author} field. {\color{red} When citing edited publications, use a script to shift the content of the {\tt author} field into the {\tt editor} field.} This enables PDF files associated with {\sc Bib}\TeX\ entries in my {\sc Bib}\TeX\ database to be placed in subdirectories in my repository of publications based on the author's (or first author's) last name.
\item Wikipedia contributors, ``TITLE OF THE ARTICLE,'' in {\it Wikipedia, The Free Encyclopedia: CATEGORY}, Wikimedia Foundation, San Francisco, CA, MONTH DATE, YEAR.
\item Wikibooks contributors, ``CHAPTER,'' in {\it TITLE OF THE BOOK}, Wikibooks: Open books for an open world, Wikimedia Foundation, San Francisco, CA, MONTH DATE, YEAR.
\item Wikibooks contributors, ``SECTION,'' in {\it CHAPTER} of {\it TITLE OF THE BOOK}, Wikibooks: Open books for an open world, Wikimedia Foundation, San Francisco, CA, MONTH DATE, YEAR.
\item Wikibooks contributors, ``TITLE OF THE BOOK,'' Wikibooks: Open books for an open world, Wikimedia Foundation, San Francisco, CA, MONTH DATE, YEAR.
\item Wikiquote contributors, ``TITLE,'' Wikiquote, Wikimedia Foundation, San Francisco, CA, MONTH DATE, YEAR.
\item Wiktionary contributors, ``TITLE,'' Wiktionary, Wikimedia Foundation, San Francisco, CA, MONTH DATE, YEAR.
\item Dictionary.com, ``WORD,'' IAC, Oakland, CA, MONTH DATE, YEAR.
\item AUTHOR, ``TITLE,'' in {\it The New York Times: The Opinion Pages: Op-Ed Contributor}, The New York Times Company, New York, NY, MONTH DATE, YEAR.
\item AUTHOR, ``QUESTION'', in {\it CATEGORY}, Quora, Inc., Palo Alto, CA, MONTH DATE, YEAR.
\item AUTHOR, Answer to ``QUESTION'', in {\it CATEGORY: QUESTION}, Quora, Inc., Palo Alto, CA, MONTH DATE, YEAR.
\item AUTHOR, ``TITLE OF POST'', in {\it BLOG TITLE}, Quora, Inc., Palo Alto, CA, MONTH DATE, YEAR.
\end{enumerate}



%%%%%%%%%%%%%%%%%%%%%%%%%%%%%%%%%%%%%%%%%
%	Common LaTeX symbols
%	This is written by Zhiyang Ong as a template for writing LaTeX symbols.

%	The MIT License (MIT)

%	Copyright (c) <2014> <Zhiyang Ong>

%	Permission is hereby granted, free of charge, to any person obtaining a copy of this software and associated documentation files (the "Software"), to deal in the Software without restriction, including without limitation the rights to use, copy, modify, merge, publish, distribute, sublicense, and/or sell copies of the Software, and to permit persons to whom the Software is furnished to do so, subject to the following conditions:

%	The above copyright notice and this permission notice shall be included in all copies or substantial portions of the Software.

%	THE SOFTWARE IS PROVIDED "AS IS", WITHOUT WARRANTY OF ANY KIND, EXPRESS OR IMPLIED, INCLUDING BUT NOT LIMITED TO THE WARRANTIES OF MERCHANTABILITY, FITNESS FOR A PARTICULAR PURPOSE AND NONINFRINGEMENT. IN NO EVENT SHALL THE AUTHORS OR COPYRIGHT HOLDERS BE LIABLE FOR ANY CLAIM, DAMAGES OR OTHER LIABILITY, WHETHER IN AN ACTION OF CONTRACT, TORT OR OTHERWISE, ARISING FROM, OUT OF OR IN CONNECTION WITH THE SOFTWARE OR THE USE OR OTHER DEALINGS IN THE SOFTWARE.

%	Email address: echo "cukj -wb- 23wU4X5M589 TROJANS cqkH wiuz2y 0f Mw Stanford" | awk '{ sub("23wU4X5M589","F.d_c_b. ") sub("Stanford","d0mA1n"); print $5, $2, $8; for (i=1; i<=1; i++) print "6\b"; print $9, $7, $6 }' | sed y/kqcbuHwM62z/gnotrzadqmC/ | tr 'q' ' ' | tr -d [:cntrl:] | tr -d 'ir' | tr y "\n"

%%%%%%%%%%%%%%%%%%%%%%%%%%%%%%%%%%%%%%%%%%%%%%



%%%%%%%%%%%%%%%%%%%%%%%%%%%%%%%%%%%%%%%%%%%
\section{Writing \LaTeX\ Symbols}
\label{sec:WritingLaTeXSymbols}


Symbols used to represent \LaTeX\ and related computer languages/technologies and concepts are: \vspace{-0.3cm}
\begin{enumerate} \itemsep -4pt
\item \LaTeX
\item \LaTeXe
\item {\sc Bib}\TeX\ (or B{\scriptsize IB}\TeX)
\item \AmS-\LaTeX
\item \MP
\item \MF
\item \texttrademark
\item \textregistered
\item To use the registered symbol as a superscript, avoid doing this in the math mode or in mathematical environments, since this will cause the registered symbol not to typeset properly. The following sequence of {\tt $\backslash$textsuperscript}{\tt $\backslash$textregistered} \LaTeX\ commands should be used instead, such as: {\it quectoSAT}\textsuperscript\textregistered\ solver. \vspace{-0.3cm}
	\begin{enumerate} \itemsep -2pt
	\item Use a backslash after it, just like the following symbols that can cause naturally occuring character space to disappear: \vspace{-0.2cm}
		\begin{enumerate} \itemsep -2pt
		\item \textsuperscript\textregistered\ needs space\dots Compared with \textsuperscript\textregistered\ needs space. \vspace{-0.1cm}
			\begin{enumerate} \itemsep -1pt
			\item nanoPlace II\textsuperscript\textregistered is far superior compared with picoPlace VI\textsuperscript\textregistered for detailed placement (without spacing).
			\item nanoPlace II\textsuperscript\textregistered\ is far superior compared with picoPlace VI\textsuperscript\textregistered\ for detailed placement (with spacing).
			\item When in doubt if a space is need, since this example does not indicate a need, use a character space anyway. It does not change the spacing by much.
			\end{enumerate}
		\item \LaTeX\ needs space\dots Compared with \LaTeX needs space.
		\item \LaTeXe\ needs space\dots Compared with \LaTeXe needs space.
		\item {\sc Bib}\TeX\ needs space\dots Compared with {\sc Bib}\TeX needs space.
		\item B{\scriptsize IB}\TeX\ needs space\dots Compared with B{\scriptsize IB}\TeX needs space.
		\item \MF\ needs space\dots Compared with \MF needs space.
		\item \MP\ needs space\dots Compared with \MP needs space.
		\item \texttrademark\ needs space\dots Compared with \texttrademark needs space.
		\item \textregistered\ needs space\dots Compared with \textregistered needs space.
		\item \copyright\ needs space\dots Compared with \copyright needs space. \vspace{-0.1cm}
			\begin{enumerate} \itemsep -1pt
			\item These symbols can be typeset in the math mode or mathematical environment.
			\item $\copyright$\ needs space\dots Compared with $\copyright$ needs space.
			\item $^{\copyright}$\ needs space\dots Compared with $^{\copyright}$ needs space.
			\end{enumerate}
		\item \textcopyright\ needs space\dots Compared with \textcopyright needs space.
		\item \textcopyleft\ needs space\dots Compared with \textcopyleft needs space.
		\item \AmS-\LaTeX\ needs space\dots Compared with \AmS-\LaTeX needs space.
		\item \officialeuro\ needs space\dots Compared with \officialeuro needs space.
		\end{enumerate}
	\item Do not use the registered and trademark symbols, $\texttrademark$\ and $\textregistered$, in the math mode or mathematical environment. Else, they would appear as other symbols.
	\end{enumerate}
\item \copyright
\item \textcopyleft
\end{enumerate}





Other symbols of interests: \vspace{-0.3cm}
\begin{enumerate} \itemsep -4pt
\item \officialeuro
\item ``$\backslash >$'': 
\item 
\end{enumerate}

















%%%%%%%%%%%%%%%%%%%%%%%%%%%%%%%%%%%%%%%%%
%	Coloring in LaTeX documents.
%	This is written by Zhiyang Ong as a template for coloring text in LaTeX.

%	The MIT License (MIT)

%	Copyright (c) <2014> <Zhiyang Ong>

%	Permission is hereby granted, free of charge, to any person obtaining a copy of this software and associated documentation files (the "Software"), to deal in the Software without restriction, including without limitation the rights to use, copy, modify, merge, publish, distribute, sublicense, and/or sell copies of the Software, and to permit persons to whom the Software is furnished to do so, subject to the following conditions:

%	The above copyright notice and this permission notice shall be included in all copies or substantial portions of the Software.

%	THE SOFTWARE IS PROVIDED "AS IS", WITHOUT WARRANTY OF ANY KIND, EXPRESS OR IMPLIED, INCLUDING BUT NOT LIMITED TO THE WARRANTIES OF MERCHANTABILITY, FITNESS FOR A PARTICULAR PURPOSE AND NONINFRINGEMENT. IN NO EVENT SHALL THE AUTHORS OR COPYRIGHT HOLDERS BE LIABLE FOR ANY CLAIM, DAMAGES OR OTHER LIABILITY, WHETHER IN AN ACTION OF CONTRACT, TORT OR OTHERWISE, ARISING FROM, OUT OF OR IN CONNECTION WITH THE SOFTWARE OR THE USE OR OTHER DEALINGS IN THE SOFTWARE.

%	Email address: echo "cukj -wb- 23wU4X5M589 TROJANS cqkH wiuz2y 0f Mw Stanford" | awk '{ sub("23wU4X5M589","F.d_c_b. ") sub("Stanford","d0mA1n"); print $5, $2, $8; print " "; for (i=1; i<=1; i++) print "6\b"; print $9, $7, $6 }' | sed y/kqcbuHwM62z/gnotrzadqmC/ | tr 'q' ' ' | tr -d "\n" | tr -d 'ir' | tr y "\n"

%%%%%%%%%%%%%%%%%%%%%%%%%%%%%%%%%%%%%%%%%%%%%%



%%%%%%%%%%%%%%%%%%%%%%%%%%%%%%%%%%%%%%%%%%%
\section{Coloring in \LaTeX}
\label{sec:ColoringinLaTeX}


Things that I can do with colors in \LaTeX: \vspace{-0.3cm}
\begin{enumerate} \itemsep -4pt
\item To change the color of the text: \vspace{-0.3cm}
	\begin{enumerate} \itemsep -2pt
	\item \textcolor{blue}{\bf TEXT}						%	\textcolor{COLOR}{\bf TEXT}
	\item \textcolor{blue}{INSERT\_STUFF\_HERE}		%	\textcolor{COLOR}{INSERT_STUFF_HERE}
	\item \colorbox{yellow}{\bf INSERT\_STUFF\_HERE}		%	\colorbox{COLOR}{\bf INSERT_STUFF_HERE}
	\end{enumerate}
\item {\bf \color{red} INSERT\_STUFF\_HERE}				%	{\bf \color{COLOR} INSERT_STUFF_HERE}
\item Common colors that I tend to use in \LaTeX: \vspace{-0.3cm}
	\begin{enumerate} \itemsep -2pt
	\item Apricot
	\item blue
	\item cyan
	\item ForestGreen
	\item green
	\item magenta
	\item RoyalBlue
	\item RubineRed
	\item yellow
	\item YellowOrange
	\end{enumerate}
\end{enumerate}








